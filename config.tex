%%%% Config version 0.1 %%%%

%Temporary packages
\usepackage{lipsum}
\usepackage{blindtext}
\usepackage{pgfornament}
\usepackage{graphicx}



%%
%% FONTS
%% using fontspec
%%
\usepackage{microtype}
\usepackage[]{fontspec}

\usepackage{ETbb}

%\setmainfont{etbb}[% in local font directory
%	Path= ./fonts/ETbb/ ,
%	UprightFont		= *-Regular.otf ,
%	BoldFont		= *-Bold.otf ,
%	ItalicFont      = *-Italic.otf ,
%	BoldItalicFont  = *-BoldItalic.otf ,
%]

%\setsansfont{NunitoSans}[% system font
%	Path= ./fonts/nunitosans/ ,
%	UprightFont		= *-Regular.ttf ,
%	BoldFont		= *-Bold.ttf ,
%	ItalicFont      = *-Italic.ttf ,
%	BoldItalicFont  = *-BoldItalic.ttf ,
%]
%
%\setmonofont{Heuristica}[% latex package
%	Path= ./fonts/heuristica/ ,
%	UprightFont		= *-Regular.otf ,
%	BoldFont		= *-Bold.otf ,
%	ItalicFont      = *-Italic.otf ,
%	BoldItalicFont  = *-BoldItalic.otf ,
%]
%
%\newfontfamily\myfont{italianno}[% Use with {\myfont <text>}
%	Path            =   ./fonts/italianno/,
%	UprightFont     =   *-Regular,
%] 

%
% PAPERSIZE AND MARGINS
% see page 1 of the memoir manual
%

\setlxvchars[\normalfont\normalsize]

\setstocksize{9in}{6in} % size of the paper
%\setstocksize{297mm}{210mm} % size of the paper
\settrimmedsize{\stockheight}{\stockwidth}{*}

%% Option 1, setting typeblock size and position
%%\setbinding{<length>}
\settypeblocksize{36\onelineskip}{\lxvchars}{*}
\setlrmargins{*}{*}{1.5}
\setulmargins{*}{*}{1.5}
\setheadfoot{3\onelineskip}{3\onelineskip}
%\setheaderspaces{3\onelineskip}{*}{*}
\setmarginnotes{17pt}{51pt}{\onelineskip}
\checkandfixthelayout[lines]

%% Option 2, setting margins
%%\setbinding{<length>}
%\setlrmarginsandblock{0.5in}{*}{1.5} 
%\setulmarginsandblock{1in}{*}{1.5} 
%\checkandfixthelayout


%%
%% PAGE STYLE
%%

\makepagestyle{mypagestyle}
%\makeheadrule{mypagestyle}{\textwidth}{\normalrulethickness}

\makepsmarks{mypagestyle}{%
	\nouppercaseheads
%	\createmark{title}{both}{nonumber}{}{}
	\createmark{chapter}{right}{shownumber}{}{. \space}
	\createplainmark{toc}{both}{\contentsname}
	\createplainmark{lof}{both}{\listfigurename}
	\createplainmark{lot}{both}{\listtablename}
	\createplainmark{bib}{both}{\bibname}
	\createplainmark{index}{both}{\indexname}
	\createplainmark{glossary}{both}{\glossaryname}
	\makeevenhead{mypagestyle}%
		{}{\normalfont\scshape\thetitle}{}
	\makeoddhead{mypagestyle}%
		{}{\normalfont\scshape\rightmark}{}
	\makeevenfoot{mypagestyle}%
		{}{\normalfont\thepage}{}
	\makeoddfoot{mypagestyle}%
		{}{\normalfont\thepage}{}
}
\pagestyle{mypagestyle}		
%
%\aliaspagestyle{plain}{mypagestyle}

%%
%% CHAPTER STYLE
%%

%\makechapterstyle{mychapterstyle}{% Default values:
%  \setlength{\beforechapskip}{50pt}
%  \renewcommand*{\chapterheadstart}{\vspace*{\beforechapskip}}
%  \renewcommand*{\chapnamefont}{\normalfont\huge\bfseries}
%  \renewcommand*{\printchaptername}{\chapnamefont \@chapapp}
%  \renewcommand*{\chapternamenum}{\space}
%  \renewcommand*{\chapnumfont}{\normalfont\huge\bfseries}
%  \renewcommand*{\printchapternum}{\chapnumfont \thechapter}
%  \setlength{\midchapskip}{20pt}
%  \renewcommand*{\afterchapternum}{\par\nobreak\vskip \midchapskip}
%  \renewcommand*{\printchapternonum}{}
%  \renewcommand*{\chaptitlefont}{\normalfont\Huge\bfseries}
%  \renewcommand*{\printchaptertitle}[1]{\chaptitlefont #1}
%  \setlength{\afterchapskip}{40pt}
%  \renewcommand*{\afterchaptertitle}{\par\nobreak\vskip \afterchapskip}
%}
%\chapterstyle{mychapterstyle}


%%
%% LOWER SECTIONS STYLE
%%

\maxsecnumdepth{part} % number sections

%% Section
%\setbeforesecskip{-3.5ex plus -1ex minus -.2ex} %If the values are negative then the first line after the heading will not be indented.
%\setsecindent{0em}
%\setsecheadstyle{\Large\bfseries}
%\setaftersecskip{2.3ex plus .2ex} %If the value is negative, then the heading will be run-in

%% Subsection
%\setbeforesubsecskip{-3.25ex plus -1ex minus -.2ex} %If the values are negative then the first line after the heading will not be indented.
%\setsubsecindent{0em}
%\setsubsecheadstyle{\large\bfseries}
%\setaftersubsecskip{1.5ex plus .2ex} %If the value is negative, then the heading will be run-in

%% Subsubsection
%\setbeforesubsubsecskip{-3.25ex plus -1ex minus -.2ex} %If the values are negative then the first line after the heading will not be indented.
%\setsubsubsecindent{0em}
%\setsubsubsecheadstyle{\bfseries}
%\setaftersubsubsecskip{1.5ex plus .2ex} %If the value is negative, then the heading will be run-in

%% Paragraph
%\setbeforeparaskip{3.25ex plus 1ex minus .2ex}
%\setparaindent{0em}
%\setparaheadstyle{\bfseries}
%\setafterparaskip{-1em} %If the value is negative, then the heading will be run-in

%% Subparagraph
%\setbeforesubparaskip{3.25ex plus 1ex minus .2ex}
%\setsubparaindent{\parindent}
%\setsubparaheadstyle{\bfseries}
%\setaftersubparaskip{-1em} %If the value is negative, then the heading will be run-in

%%
%% TABLE OF CONTENTS
%%

\maxtocdepth{section} % include sections
%\renewcommand{\cftchapterpagefont}{}
%\renewcommand{\cftchapterfont}{} 

%%
%% LETTRINE CONFIG
%%

\usepackage{lettrine}
\usepackage{Zallman} %https://ftp.acc.umu.se/mirror/CTAN/macros/latex/contrib/lettrine/doc/lettrine.pdf
% https://ftp.acc.umu.se/mirror/CTAN/macros/latex/contrib/cfr-initials/cfr-initials.pdf
%
\renewcommand{\LettrineTextFont}{\rmfamily}
%\LettrineRealHeighttrue
%\newcommand{\mylettr}[1]{\lettrine{\zall{#1}}} % lettrine with initial letter
%\newcommand{\mylettr}[1]{#1} % Turn off lettrine
\newcommand{\mylettr}[1]{\lettrine[lines=2]{#1}} % lettrine without initial letter

%%
%% CUSTOM COMMANDS
%%

%% Asterism
% <code for asterism>


%% \versopic{<picturename>}{<width>}{<undertext>}
%% picture centered on the next verso page, intended just before new chapter
\newcommand{\versopic}[3]{%
\cleartoverso
\thispagestyle{plain}

\begin{figure}[p]
	\centering
	\includegraphics[width=0.7\linewidth]{images/person1}\par
	\vspace{10pt}
	\centerline{Förnamn Efternamn, årtal}\par
	\label{fig:test}
\end{figure}

}

%% 
%% see page  of the memoir manual
%%


%% 
%% see page  of the memoir manual
%%


%% 
%% see page  of the memoir manual
%%