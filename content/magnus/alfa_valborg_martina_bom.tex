\chapter{Alfa Valborg Martina Bom}
\label{alfa_valborg_martina_bom}
Född 18/12 1921 i Fågelfors.Föräldrarna hette Arthur och Anna Bom. Hon har 2 yngre syskon.Första åren bor familjen på övervåningen i ett gult hus bredvid mejeriet i Fågelfors. 1937-38 bygger Arthur ett rött trähus. Det är det huset som ligger närmast Lindéns gula villa. 
Valborg går 6 år i skolan. 1937 har hon gått en 6 veckors kurs i Kalmar Läns Södra landstings skolkök. I kursen ingår enklare matlagning, bakning, kortfattad födoämneslära samt ordning och renlighet.Från sommaren 1939 arbetar hon som köksbiträde i 2 månader och som barnhusa i 1 år och 6 månader hos familjen Ekströmer på deras herrgård i Fågelfors. Tiden på herrgården är Valborg väldigt stolt över hela sitt liv. Det var nämligen bara flickor som kom från ”fina” familjer som fick arbete där. 1941 tjänstgör hon på barnkolonien Walldahemmet i Halland mellan 15 juni och 15 augusti. 15 juni 1942 och 15 januari 1943 arbetar hon som hembiträde hos Selma Henriksson i Djursholm. 1946 går hon en kurs i vävning i regi av Södra Kalmar Läns Hembygdsförening.

Valborgs syskon:

Elin,1926, gifter sig med Ruben Franzén 1961 i prästgården i Högsby.  Elin bär en vit, lång klänning, brudkrona och har röda rosor i sin brudbukett. Bröllopsmiddag intas på Hotell Morén. De träffades redan 1945, men av olika anledningar dröjde bröllopet. Elin utbildar sig till hemvårdarinna. Innan giftermålet bor och arbetar  hon i Mönsterås och Kalmar. Efter vigseln köper paret ett hus i Nybro. Elins sista arbete blir på Madesjö skola. Ruben är ingenjör och vägmästare. 

Barndomsminne: Elin och Valborg sover i en utdragssoffa. En kväll blir de törstiga och stiger upp för att dricka vatten i köket. Direkt efter att de är ur bädden så trillar väggklockan ner och slår upp ett stort märke i soffan och hamnar precis där flickorna har legat. Där var nog en änglavakt inblandad.
Kökssoffan står nu i Kvillehult och klockan hänger på väggen här i köket på Åsvägen. Klockan har en egen historia, släkt från Amerika har haft den med sig vid ett Sverigebesök. Det fattas en målad bild i ”glasfönstret”. Den fick några släktingar som minne av dem som hade skickat klockan en gång i tiden.

Mauritz, 1930, gifter sig med Waldy Johansson i Kalmar slottkyrka. Mottagning efteråt i deras bostad på Storgatan i Högsby. De får två barn, Anders och Birgitta.
Ett starkt minne från den 28 januari 1935: Mauritz sitter på sin mammas arm och de står och tittar på den stora branden vid Fogelfors Bruk. Han minns att hans pappa var med och försökte släcka elden och hur han var klädd i en stor tjock päls som av kyla och vattensläckning står för sig själv när han tar av sig den.

Sigurd och Valborg träffades på ett lite märkligt sätt genom olika personer i bägges närhet. 
Valborgs farmor, Selma Åberg Bom bor på Bruksgatan i Fågelfors. När Arthurs bror Henry har gift sig med sin Agda, så får de bygga sitt hus på samma tomt. 
Nu kommer det märkliga. Henry är Valborgs farbror och Agda är Sigurds syster. I och med detta så umgås bägge familjerna flitigt men det dröjer flera år innan Sigurd och Valborg upptäcker varandra. 
Dvs. Sigurd gör nog det med en gång men Valborg är förlovad med en annan. Fästmannen flyttar till Stockholm och det är väl meningen att hon ska komma efter men under tiden träffar han en annan flicka och förlovningen bryts.
Tiden går och en dag får Valborg ett brev från Sigurd. Han skriver om sina känslor för henne och frågar om det finns någon chans att det ska bli de två. Han vill dessutom att hon ska bränna brevet när hon har läst det. Det gör hon ju naturligtvis inte utan hon gömmer det i en kokbok. Brevet hittar vi av en slump 50 år senare när deras hem säljs och alla saker gås igenom och delas upp.
De gifter sig 24 maj 1947 i Fågelfors prästgård. Mottagning hos hennes föräldrar i det röda huset. Hela släkten är bjudna. Nils Forsblom som är gift med Valborgs faster Elin har tagit med sig fyrverkerier till bröllopet. När han kastar upp en pjäs i luften så fastnar den i en tall som börjar brinna, men allt slutar bra, alla hjälps åt att släcka. 
Det är fortfarande svårt att få köpa allt som behövs, så släkten bidrar med kuponger till bl.a mat. Valborgs faster Anna, som bor i Kalmar, kommer med en stor krokan som är 1 meter hög. Den översta delen, själva kronan på verket, sparar Valborg som ett minne. Sigurds systrar Astrid och Aina har bakat smörbakelser och kringlor i sitt bageri.
Sigurd har byggt ett hus på Parkvägen i Fågelfors som får heta Björkhaga och där flyttar brudparet in och trivs riktigt bra. Snett mittemot har Sigurds bror Erik byggt ett  hus till sin familj.  .
Men Sigurd kan inte släppa sin dröm om en egen gård och efter ett par år kommer tillfället som gör att han kan förverkliga drömmen. Valborg är mer tveksam, gården med tillhörande skog är nergången och det krävs en stor renovering för att kunna bo där. Sigurd vill så gärna ha gården och ber henne att ge det en chans, han lovar henne att om hon efter några år inte vill bo kvar så ska de flytta till ett bättre hus. När åren har gått kan Valborg inte tänka sig att bo på något annat ställe, så de stannar kvar.

Sigurd och Valborg köpte Kvillehult 1951 av Wilhelm och Gerda Jonsson för 32 000 kr. I köpet ingick hus, ladugård, skog, åkrar och en fotogenlampa som numera hänger över köksbordet här på Åsvägen. Det krävdes många och omfattande renoveringar innan familjen kunde flytta in. Det inköptes traktor av märket Ferguson, kor, grisar och höns. 
På 1950-talet var Blankan en levande och folkrik by med många  barnfamiljer. Det fanns en affär som ägdes av Evert och Nanna Gahne, snickeriverkstad, kvarn, kraftstation och en smedja. 