\chapter{Hultsnäs, Mörlunda}
Hultsnäs hittar vi i kyrkböckerna först år 1814.
Nuvarande hus är nog inte från den tiden men genom att följa släkten bakåt kan vi se att huset är byggt före år 1860 för då gifter sig Caisa-Lisa Magnusdotter född 1839 i Näshult med Carl-Johan Pettersson född 1836 i Lilla Boda med varandra och flyttar till Hultsnäs.

Här lever de sina liv och får 10 barn mellan 1861 och 1881 varav 9 uppnår vuxen ålder. 
Som nummer 4 föds Oskar år 1867.
Oskar växer upp och träffar sin blivande hustru från granngården Lilla Bölö, hon är soldatdotter och heter Maria Kristina Sköld och är född 1873.
De gifter sig i mars 1897 och första barnet Einar föds 9 månader senare. 
Under åren 1897 och fram till 1918 får de 10 barn varav 8 når vuxen ålder och sist i raden föds Erik.I Hultsnäs tar de även hand om Oskars föräldrar under många år och Lena Stina Henriksdotter till Hultsnäs.

I april 1923 flyttar hela familjen till Fågelfors och köper gården Hässlås. 
Man kan tänka sig att Kristina som var den enda av de 6 vuxna syskonen som inte emigrerade till Amerika fick ut arv från sina föräldrar som hade avlidit några år tidigare.

Oskars far Carl-Johan som kom från Lilla Boda tillhör en familj som bott i Lilla Boda sedan början på 1700-talet. 
De första vi hittar i Lilla Boda är Nils Göransson och Maria Nilsdotter vars son Nils Nilsson, född 1732, tar över gården och gifter sig med Lisa Nilsdotter. 
Deras son Carl Nilsson, född 1775, tar sedan över Lilla Boda tillsammans med sin hustru Britta Jonsdotter. 
Gården fortsätter i släkten när deras dotter Stina Kaisa Karlsdotter tar över gården tillsammans med Peter Karlsson från Borg i Mörlunda. 
Med dessa två är vi tillbaka till Carl-Johan som är deras son och Oskars pappa.