\chapter{Oskar Sigurd Mauritz Karlsson}
\label{oskar_sigurd_mauritz_karlsson}
Sigurd föddes 12 februari 1910, han får namnen Oskar Sigurd Mauritz av sina föräldrar Oskar och Kristina Karlsson. Oskar arbetar på Ryningsnäs Gods som arbetskarl och i hans sysslor ingår också att vara roddare. Ofta inbjöds gäster att deltaga i jakter såsom harjakt och tjäderjakt, men det var framför allt andjakten i den då utmärkta Ryningen som var mest eftertraktad. Då fick gårdens arbetare ställa upp som roddare och upplockare av bytet. De skulle också gående genom vassruggarna få änderna att lyfta och helst då i riktning mot den plats där skytten befann sig. Vid en sådan här andjakt sköts de första dagarna de var lovliga för jakt, inte mindre än 600 änder av ett inbjudet 3-mannalag. Familjen bor i Hultsnäs i Mörlunda socken. I familjen ingår även Sigurds morföräldrar Carl Fredrik och Lena Christina Sköld. Sigurds första barndomsminne är när hans morfar har dött 1912. Av någon anledning så dröjer det innan han körs till Mörlunda för begravning så han läggs i en kista och ställs ut i ett uthus på gården. Sigurd är då ungefär 2 ½ år.

Ett annat minne rör en gammal gumma som familjen bor granne med. Hon är fattig och bor ensam i en liten stuga som ligger ett par hundra meter från Hultsnäs. Sigurd och hans mamma gick ofta dit med mat till henne.
När Sigurds mormor dör 1918 flyttar familjen till Lilla Boda där sista barnet föds. Nu är de 8 syskon. Ytterligare två barn har dött vid födseln 1899 och 1918.
Troligen är Lilla Boda en släktgård på Oskars sida.
Runt 1930 går flytten till Hässlås, Fågelfors. Familjen har utökats med två barn, Rut som faktiskt var Oskar och Kristinas barnbarn( hennes mamma har dött och pappan kan inte ta hand om sitt lilla barn) och en fosterpojke som heter Lennart Gustafsson.

Oskar blir dement de sista åren, det är svårt att ta med honom på t.ex släktkalas så han får stanna hemma. För att han ska ha något att göra så får han en liten uppgift som de vet att han klarar av. En gång sa hans son att han kunde bära in ved i köket medans alla var borta. Det gjorde Oskar med besked, han fyllde hela köket tills det inte fick rum ett vedträ till. 
Valborg har berättat om ett besök i Hässlås, lilla Ann-Mari  kryper omkring på golvet i lugn och ro. Plötsligt hör Valborg att Ann-Mari gråter och springer in i rummet där även Oskar befinner sig. Ann-Mari sitter gråtande under ett bord med alla stolar ordentligt inkörda. Oskar är väldigt nöjd, han talar stolt om att han har stängt in alla kalvar så de inte smiter sin väg.
Oskar dör 1949, hans fru Kristina 1959. Hon dog i sin egen säng och det sista gör är att sätta sig upp  och titta ut genom fönstret samtidigt som hon säger
Sigurd har nu tagit körkort, han har berättat hur det gick till. Efter några körlektioner i Oskarshamn så är det dags för uppkörning, det går bra. Teoriprovet består av två frågor, den ene var så här: Får man köra bil om man har druckit sprit? Sigurd svarar nej, det får man inte. Då säger körskoleläraren: Jo, lite får man ta. Körkortet utfärdas 21/4 1931.

Sigurd hinner med många yrken i sin karriär. Bland de första arbeten han hade var som snickare på Nya Fabriken i Fågelfors. Samtidigt hade han extrainkomst hemma på gården i Hässlås, en silverrävsfarm. Skinnen såldes och det blev lite extra pengar i mitten av 1930-talet. Sedan arbetade han som murare hos Holmström i Fågelfors. Under andra världskriget hade han kolugnar i Hässlås och sålde kol till Försvaret. Sigurd berättade hur Fogelfors bruk såg honom som konkurrent och försökte svälta ut honom genom att köpa in all ved till överpris. Men en dag kom folk från bränslekommissionen och undrade varför hans kolugn stod stilla när behovet av kol var stort, han berättade som det var, varvid kommissionen löste in brukets ved som Sigurd fick ta över. Inköparna på bruket stod där med lång näsa. Efter kriget köpte han en lastbil och körde grus, samt använde den till att hämta mjölk hos bönder och köra den till mejeri.

\section{Sigurds syskon}

Einar, 1879-1971, gifter sig med Naemi  Karlsson och får två barn, Berit och Alf. Han tar över Hässlås efter sina föräldrar. Familjen flyttar sedan till Hultsfred.

Astrid, 1900-1960, gifter sig med Gunnar Thyrén och får en son, Enar. Hon har eget företag i Fågelfors, kafé och bageri som är beläget i huset bredvid Filadelfiahuset. 

Elna, 1902-1930, gifter sig med Karl Gunnar Karlsson och får en dotter, Rut. Elna dör när Rut är 1½ år, pappan kan inte ta hand om henne utan hon växer upp i Hässlås hos sina morföräldrar.

Agda, 1904-1990, gifter sig med Henry Bohm och får två barn, Gerd och Roland. Agda var en händig person, hon sydde både till familjen och andra, målade om och tapetserade när det så behövdes i hemmet. Hon var duktig vid vävstolen, kunde väva många invecklade mönster. Henry var en spjuver, han tyckte om att skoja och tävla om de mest besynnerliga saker. Han hade Fågelforsrekord i att kunna joja längst, han höll på hela natten och vann till slut. Tävlingen om vem som kunde äta mest senap vann han naturligtvis.

Märta, 1907-1978, var särlingen i familjen. Hon reser till Stockholm och söker lyckan. Hon gifter sig med Herman som senare visar sig vara oärlig. Han säljer gödsel, men det är bara vatten och för det får han sitta av sitt straff på Långholmen. Innan dess flödade pengarna, paret bjuder Märtas föräldrar till Stockholm och tar med dem till Berns Salonger. Herman dör ganska tidigt och Märta berättar att Herman går igen och visar sig som en liten figur i deras hem.Efter detta har Märta tre förhållanden. Yngve, Gustav och Lennart. När Märta dör på 1970-talet åker Sigurd och hans bror Erik till bodelningen i Stockholm. Sigurd får bl.a ärva väggklockan av guld. 

Aina, 1913-1991, gifter sig med John Enocksson och får en dotter, Ann-Britt. Aina arbetar en tid hos Astrid på kafét.

Erik, 1918-1980, gifter sig med Eva Petersson och får tre barn, Monika, Inga-Lill och Carola. Samtidigt som Sigurd bygger huset Björkhaga på Parkvägen så bygger Erik sitt hus mittemot Sigurds. Kort efter att Sigurd med familj flyttar till Kvillehult så köper Erik en gård som heter Rydet och ligger på vägen till Virserum. Den äldsta dottern, Monika, vill se världen och hamnar i Argentina där hon bor med sin familj i många år. Inga-Lill gifter sig med Sixten Jacobsson och får en son, Mikael. Hon blir änka tidigt. Bor i Virserum. Carola gifter sig med Jan Strömberg och får tre barn. Efter Eriks död tar Carola med familj över Rydet och flyttar dit.

Fosterbror Lennart Gustafsson gifter sig med Berit och får tre barn. Familjen bor i Fågelfors och räknas som släkt hela tiden.

Arvegods från Sigurd och Valborg: det blå stora skåpet och väggklockan kommer från Sigurds föräldrahem. Vid bodelningen efter hans mor,1959, så var det ingen som ville ha dessa saker. Efter en stund säger någon av syskonen ” Ge dem till pojken ” och så blev det, Magnus fick dem. Örnen högst upp på klockan saknades. Många år senare så snidade Lindström i Fågelfors en ny. Magnus fick igång klockan och den gick så fint tills en natt då vi vaknade av att den slog och slog. När vi hade räknat till över 60 slag så fick det vara nog och Magnus stannade den. Sen dess har vi inte fått igång den mer. Det blå skåpet stod en gång i tiden i Hultsnäs ( Sigurds barndomshem ) innan det följde med till Hässlås.
Guldklockan kommer från Sigurds syster Märta. Skrivbordet som Magnus har nu var Sigurd och Valborgs skrivbord i Kvillehult. Fia-spelet köpte Valborg på en marknad i Virserum, på spelet har Pälle Näver skrivit en dikt, ”Gick här och frös, då kom en tös, solig och grann, och kylan försvann”.
Den blå mjölkpallen och mjölkkannan har varit Valborgs farmor. Den bruna kaffekvarnen är troligen hennes också samt den lilla kaffekokaren av koppar. I Kvillehult i hallen uppe står en grön sammetssoffa som stod i Fågelfors. Selma Åberg köpte den i Vetlanda. Den gröna kaffekvarnen kommer från Sigurds föräldrar.
Tavlan med fåglarna
Magnus fick ett fickur av sin pappa när han fyllde 25 år. När Peter fyllde 25 år fick han den av Magnus.