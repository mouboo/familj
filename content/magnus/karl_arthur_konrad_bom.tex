\chapter{Karl Arthur Konrad Bom}
\label{karl_arthur_konrad_bom}
Magnus morfar, Arthur Bom född 1890-08-12 i Fågelfors arbetade som skogvaktare, brädgårdsförman på snickerifabriken  och till sist kalkylator vid Fogelfors Bruk.

Föddes 1890-08-12 i Sjötorp i Fågelfors. Föräldrarna var Carl August Bom och Selma Terese Åberg. Namnet Bom kommer från Arthurs farfars farfar Nils Persson som var född i Kråksmåla och blev soldat i Högsby under Bondeberga nummer 87, han fick då namnet Bom.

Namnet Åberg är ett vanligt namn i Fågelfors med omnejd och de flesta av dessa har nog gemensamma förfäder. Åbergsläktet i Fågelfors är en smedsläkt som man kan härleda tillbaka till en Nils Larsson född i Rumskulla och som sedan fick namnet Åberg. Han blev smedmästare vid Fredriksfors bruk i Döderhult. När Fredriksfors järnbruk lades ner flyttades verksamheten till Fågelfors och släkten Åberg följde med dit. Går man på kyrkogården i Fågelfors idag hittar man namnet  Åberg på många stenar.

Artur var  Carl August och Selmas första barn men snart kom flera syskon Gunnar, Anna, Elin och Henry.  I Juni 1903 när Henry var knappt ett år dör Carl August och Selma står nu ensam med 5 barn.  I början av 1890-talet hade familjen flyttat till ett hus i Fågelfors som ägdes av bruket och nu när Carl August var död meddelade brukspatron Ekströmmer att de var tvungna att flytta före ett visst datum om de inte kunde köpa huset. Fyra av Selmas syskon och tre av Carl Augusts syskon hade utvandrat till Amerika och när de fick höra om sin systers dilemma sände de hem pengar och Selma kunde stega in till Brukspatronen och meddela att hon hade pengar till att köpa loss huset om nu Brukspatron stod vid sitt ord.

Så här såg köpekontraktet ut. [bild]

Arthur och hans bror Gunnar fick redan vid 12 och 13 års ålder börja arbeta vid Fogelfors Bruk, här kan vi se utdrag ur löneboken för några månader under 1903. Vi kan se att Arthur tjänade 65 öre om dagen och Gunnar 10 öre för varje kväll efter skolan.

[bild]

Gunnar beslutade sig för att fara till släkten i Amerika och vid 19 års ålder startade hans färd. När han kom fram skrev han ett brev hem och berättade om resan och mötet med sin släkt. Originalbrevet är skriven med en handstil som många har svårt att läsa idag, därför har vi skrivit av brevet vilket återges här.

[bild]

Arturs syster Anna fick jobb som hembiträde hos Konsul Jansson i Kalmar. Hon förblev ogift och man kan anta att hon betydde mycket för honom eftersom hon fanns med i hans testamente och fick för den tiden en ganska stor summa pengar.
Plats för testamente

[bild]

Den sista i barnaskaran Henry blev Fågelfors trogen, arbetade på bruket, gifte sig och byggde hus på samma tomt som Selmas hus stod på.

Arthur och hans syskon växte upp under knappa förhållande där de lärde sig att ta ansvar.

[bild Arthur]

[bild Carl August]

[bild Selma]

