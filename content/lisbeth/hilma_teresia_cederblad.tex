\chapter{Hilma Teresia Cederblad}
\label{hilma_teresia_cederblad}
Född Cesar, den 27/7 1890 i Bärbäcksbron, Alseda.

Gick i Alseda kyrkskola, och konfirmerades i Alseda kyrka.
Därefter blev det att ta anställning som piga på olika gårdar.
Hon arbetade på Vallsjö gård, Sävsjö, Nyaby Berg, Slättåkra, hos Carl Zakrisson Repperda, och hos Emil Gustavsson Repperda Sunegård.

Under tiden i Repperda blev hon medlem i Logen Ädelfors Framtid av NOV, som fanns i Ädelfors i början av 1900-talet. Medlem var också Anders Cederblad från Ädelfors, som hon gifte sig med i maj 1913. Kanske var det på logemötena som de lärde känna varandra.
Efter giftemålet blev hon som var vanligt på den tiden hemmafru, skötte hem och barn. Skaffade sig symaskin, och var duktig att sy både till sig själv och barnen. Lånade ibland vävstol och vävde trasmattor till hemmet.

Under några år i början av äktenskapet hade man också hushållsgris som slaktades strax före jul. Att ta hand som slakt hade hon ju lärt sig på gårdarna hon varit på. Så det blev blodpalt och pölsa och flera sorters korv, pressylta och rullsylta och julskinka lagades till. Fläsket saltades ned i träkar, för kommande behov.

Blev änka i juli 1979, och råkade våren 1971 ut för en olyckshändelse (lårbensbrott), och kom efter lasarettsvistelsen i Eksjö, som konvalecent till Österliden i Holsbybrunn. På egen begäran fick hon sedan stanna kvar där till sommaren 1981, då hon insjuknade och fick komma till Vetlanda Sjukstuga, där hon avled den 12/10 1981.

\begin{flushleft}
Barn:\\ 
Georg Bernhard Fritiof född den 24/9 1913\\
Bernt Gösta Henry född den 29/6 1916\\
Berta Viola född den 19/10 1918, död den 14/2 1919\\
Viva Gunhild Maria född den 9/11 1920\\


Hilmas syskon:

Karl Johan Levi Caesar född 16/5 1872 emigrerar till Amerika.
Frida Lovisa Josefina Caesar född 15/5 1875 emigrerar till Amerika.
Gustav Alfred Irenius Caesar född 15/12 1877 emigrerar till Amerika.
Helga Emilia Caesar född 17/11 1880. Gift med snickaremästare Sven Lindberg i Slättåkra. Han tillverkade hyvelbänkar och slöjdbänkar.
Hjalmar Fredrik Caesar född 31/10 1883. Smed. Gift med Gulli Maria Teresia Franzén.
Emma Vicktoria Caeasar född 16/1 1887. Gift med August Elof Karlsson 1916, innan dess har hon arbetat som tjänarinna  på Vagnhester Kåragård.
\end{flushleft}
