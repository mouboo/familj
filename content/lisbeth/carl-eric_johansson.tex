\chapter{Carl-Eric Johansson}
\label{carl-eric_johansson}
Född 19 mars 1921 i Åseda. Föräldrarna hette Carl och Linnéa Johansson.  

Carl-Eric hyser en dröm att bli postiljon (brevbärare) men efter konfirmationen börjar han arbeta på ett plåtslageri i Åseda. Den anställningen blir inte så lång, han har dessvärre väldigt svårt för höjder och att lägga plåttak är inte att tänka på. Första arbetet är på ett kyrktak, så den karriären blev kort. Dessutom trycker  Carl på att han ska lära sig till skräddare och ta över rörelsen så småningom. Carl är oerhört duktig i sin yrkesroll, allt ska vara perfekt innan han godkänner något. C-E har berättat att om det så var bara ett stygn som inte var som det skulle i Carls ögon så tog han tag i sömmen och rev isär den. Det måste ha varit en svår lärotid, tills C-E själv blev skräddare. Han fortsatte att arbeta för sin far ända till i början av 1960-talet då han fick ny anställning på Garpens konfektionsfabrik. På sin fritid ägnade sig C-E åt fotboll, ishockey, bandy och tennis. På den här tiden (1930,-40-50 och -60 talen) fanns ingen särskild anställd som klippte gräset på fotbollsplanen eller på vintern spolade is i ishockeyrinken m.m. Utan det var de aktiva inom vardera sport som fick hjälpa till med det. Det var även problem med utrustningen när pengarna inte räckte till nyköp. C-E hade t.ex inte råd att köpa sig ishockeybenskydd så han sydde ett par istället.

Carl-Erics syskon:

Maj-Britt, 1918, gift med Torsten Johansson. En son, Roland.

Ann-Margret, 1920, gift med Folke Johansson. En son, Lennart.

Gullan, 1926, gift med Lars Stéen. Tre döttrar, Margareta och tvillingarna Kristina och Katarina.