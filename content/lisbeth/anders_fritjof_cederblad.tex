\chapter{Anders Fritjof Cederblad}
\label{anders_fritjof_cederblad}
Född 21/1 1888 i Kibbe, nuv.~Ädelfors. Gick i Kibbe skola och konfirmerades i Alseda kyrka den 24 april 1903. Gifte sig i maj 1913 med smeddottern Hilma Teresia Cecar, från Bärbäcksbron i Alseda.
Gjorde värnplikten vid Kalmar Regemente på Hultsfreds slätt, 30/4 - 27/12 1910 = 240 dagar. Arbetade som smältare vid masugnen i Ädelfors 1914-1918, och förlorade ena ögat vid olycksfall i arbetet. Han fick stänk från smältan i ögat, och synen gick inte att rädda. Arbetsförmågan ansågs nedsatt med 25\%, och han tillerkändes en livränta på 300 kr per år, som utbetalades med 75 kr i kvartalet. Prästbevis att han fortfarande fanns, måste insändas 2 gånger per år.

1919 upphörde verksamheten vid masugnen i Ädelfors. Gården Germunderyds Ebbagård som varit i Brukets ägo, såldes till AB Kalmar Kol och Trävaruaffär, som genast satte igång med avverkning på den skogrika gården. Då blev det skogs och såberksarbete, och även kolning förekom. Senare blev han kantare vid sågen i Ädelfors.

1932 lade Kalmar Kol och Trävaruaffär ned verksamheten, och sålde Germunderyds Ebbagård till Domänverket.

1931 byggde han tillsammans med systern Anna Gustava eget hem i Ädelfors. Köpte ett gammalt timmerhus av Emil Edborg i Repperda, som fick utgöra stommen i det nya huset. Det var dock längre än det nya skulle bli. Därför tog man bort en bit på mitten och fogade ihop båda halvorna. Anna Gustava satsade 2000 kr mot löfte att få disponera våningen om ett rum och kök, så länge hon levde.

Själv tog han ett inteckningslån på 2000 kr och med mycket eget arbete gick det att klara av det hela. Huset fick namnet Cederslund. En bit in på 1930-talet blev det dåliga tider, med stor arbetslöshet, och svårt att få arbete. Han jobbade med stenröjning på åkrar i Möcklarp, var byggnadssnickare, fick jobb som kantare vid olika ambulerande sågverkm och var kantare vid Carl Blomstrands såg i Holsbybrunn ett par vintrar.

Under krigsåren 1940-1945 högg han meterved på Domänverkets gård i Germunderyd.
Sommaren 1945 började han arbeta på Bröderna Granlunds Båtbyggeri i Ädelfors, och stannade där till semestern 1958, samma år som han fyllde 70 år.

Avled efter en tids sjukdom, på Eksjö Lasarett den 28/7 1970.

--- --- ---

Lisbeths morfar, Anders Fritiof Cederblad, född 21/10 1888 i Kibbe, nuvarande Ädelfors. Gick i Kibbe skola och konfirmerades i Alseda kyrka den 24/4 1903. Gifte sig 23 maj 1913 med smeddottern Hilma Teresia Caesar från Bärbäcksbron i Alseda.
Gjorde värnplikten vid Kalmar Regemente på Hultsfreds slätt, 30/4 – 27/12 =240 dagar.
Arbetade som smältare vid masugnen i Ädelfors 1914 – 1918 där han förlorade sitt ena öga vid  olycksfall i arbetet. Han fick ett stänk från smältan i ögat och synen gick ej att rädda. Arbetsförmågan ansågs nedsatt till 25\% och han tillerkändes en livränta på 300 kr/år som utbetalades med 75 kr i kvartalet. Prästbevis att han fortfarande fanns måste skickas in 2 ggr/år.
1919 upphörde verksamheten vid masugnen i Ädelfors.
Gården Germunderyds Ebbagård som varit i Brukets ägo såldes till AB Kalmar Kol och Trävaruaffär, som genast satte igång med avverkning på den skogrika gården. Då blev det skogs-och sågverksarbete och även kolning förekom. Anders fick arbete där och senare blev han kantare vid sågen i Ädelfors.
1932 lade Kalmar Kol och Trävaruaffär ned verksamheten och sålde Germunderyds Ebbagård till Domänverket.
En bit in på 1930-talet blev det dåliga tider, med stor arbetslöshet och därmed svårt att få ett arbete. Anders arbetar med stenröjning på åkrar i Möcklarp, byggnadssnickare, fick arbete som kantare på olika sågverk bl.a vid Carl Blomstrands såg i Holsbybrunn ett par vintrar.
Under krigsåren 1940 – 1945 högg han meterved på Domänverkets gård i Germunderyd. Sommaren 1945 började han arbeta på Bröderna Granlunds Båtbyggeri i Ädelfors där han stannade till semestern 1958, samma år som han fyllde 70 år.
Georg har intervjuat Anders om hans yrkesliv, det finns sparat på en cd som jag har. I bakgrunden( om man lyssnar noga ) kan man höra mormor Hilmas röst ibland och ljudet från en väggklocka.

1931 köpte Anders ett gammalt timmerhus av Emil Edborg i Repperda, som fick utgöra stommen i det nya huset. Det var dock längre än det nya skulle bli så därför tog man bort en bit på mitten och fogade ihop de båda halvorna. Husdelarna transporterades med häst och vagn till Ädelfors. Anders tog ledigt ett par sommarmånader och med hjälp av två snickare så byggdes det nya huset.
Anders tog ett inteckningslån på 2000 kr och hans syster Anna Cederblad satsade 2000 kr mot löfte att få disponera övervåningen om ett rum och kök så länge hon levde.
Huset fick namnet Cederslund.

Anders syskon:

Matilda Kristina Cederblad, f. 1865-1939.. Flyttar till Vidbo, Stockholm 1881. Gift med August Leonard Ludvig Lundberg. Han äger ett plåtslageri. Bor i Upplands Väsby med man och fosterdotter Ingbritt Maria Eleonora Cederblad, vars biologiska mor var Matildas syster Hilda.

Maria Lovisa, f. 1867-1914, flyttar till St. Jakobs församling i Stockholm 1885. Arbetar som tjänarinna på Elfsjö gård, Stockholm.

Anna, f. 1869-1938, arbetade på Upplanda Herrgård i Vetlanda hos familjen Adlercreutz, hon hade anställning där när den stora skandalen i familjen hände. Dottern i familjen var gift med löjtnant Sixten Sparre och de hade två barn, men han förälskade sig i  lindanserskan Elvira Madigan och rymde med henne till Danmark. När deras pengar tog slut, så såg de ingen annan utgång än döden. Sparre sköt dem båda med sin revolver. Denna händelse inträffar i slutet av 1880-talet så det blev en väldig uppståndelse. Dramat har även blivit film. Skrivbordet och det fina blå fatet som jag har nu  kommer från Annas ägodelar. Klockan som ligger i sin originalask kommer troligtvis från Anna också.


Karolina Josefina, f. 1871-1965, flyttar till Stockholm och får arbete hos Grosshandlare Brisman och hans familj. Senare bor hon hos en dotter i samma familj.
Efter Karolinas död, så åker Vivas kusiner till Stockholm i en hyrd lastbil och hämtar hennes ägodelar. Senare hålls en auktion i Vetlanda.

Carl-Oskar, f. 1874-1956, maskinist. Gift med Anna Lovisa Carlsdotter, 7 barn.

Johan Henrik, f. 1876-1901.

Johanna(Hanna) Augusta, f. 1878. Gift med Carl Fredrik Köhler, 5 barn. De tre första dör unga i Sverige. Familjen emigrerar till Amerika.

Gustav Adolf, f. 1881-1963, maskinist. Gift med Sigrid Virena Emilia Reik, 6 barn.

Claes Ferdinand, f. 1883-1884.

Hilda Alma Elisabet, f. 1886-1977. Ogift, 2 barn. Ett av barnen blir fosterdotter hos Hildas syster Matilda Kritina. Hilda flyttar runt ofta, först till Eksjö Stadsförsamling 1906, sen till Eksjö Ränneslätt 1907. Nästa anhalt är Barnarps prästgård 1908,  22/10 1909 går färden till Säby, kvarteret Berget 102. Där blir hon gravid, åker hem till Ädelfors och föder en son, Carl Ragnar 1910, åker tillbaka till Säby med pojken. 12/5 1911 återfinns Hilda med son i en lista med rubriken
”Personer utan fast bostad”. Hon  flyttar till Hammarby 5/3 1913 och sonen kommer efter 16/1 1914. Ännu en gång blir hon gravid  och får en dotter, Ingbritt Maria Cecilia 10/2 1915. Hilda är inte gift, men en man som heter Karl August Lundgren ”låter anteckna sig som barnets fader”. Han är rättare i Björkboda.
Hilda arbetar som tvätterska i ett tvätteri som hon har ihop med syster Matilda. Detta har de i sitt hem och tvättar i bäcken som rinner förbi i närheten av huset. De hämtar tvätt hos ”finare” familjer och lämnar tillbaka den tvättad, struken och manglad.