\chapter{Johan August Cecar}
\label{johan_august_cecar}
Född 31/3 1846, Qills Pappersbruk Ökna.
Var smed och bodde i Bärbäcksbron, där han också hade sin smedja.
Gifte sig 1/12 1871 med Johanna Lovisa Johansdotter , från Godanstorp Alseda. Bodde under uppväxttiden på ett ställe under Trollhälla, där fadern enligt hörsägen hade en smedja.

1876 köpte han stugan på Broatorpet (Bärbäcksbron) som låg på Trollhälla gårds marker. Han hade tidigare fått löfte att bygga en stuga i närheten, men då torparen avflyttat och den stugan blev ledig, avstod han från att bygga. Priset för Bärbäcksbron blev 150 kr. Trollhälla gård ägdes av Alseda sockenm och där fanns även socknens fattiggård. Troligen byggde han i samband med detta smejdan nere vid brofästet till bron över Emån. 

Förutom det vanliga “bondsmidet” med reparationer och skoning av hästar och dyligt, tillverkade han diverse verktyg som potatishackor, lövhackar och “riskar” som han sedan sålde på torget i Vetlanda. Åkte då med någon bonde som hade egna ärenden till torget. Hans dotter Hilma har berättat att när han kom hem från torgresan så tömde han ut smålandspungen på spiselhällen, och räknade noggrant ihop dagskassan. Det var väl inte alltid så lätt att få slantarna att räcka till och han beklagade sig ibland. ``En är så gällskyldig så en vet inte ut vart in en ska vända sig.'' Gällskyldig betyder skuldsatt.

Barn: Karl, Frida, Gustav, Helga, Hjalmar, Emma, och Hilma.

