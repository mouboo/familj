\chapter{Lisbeth Gunvor Maria Johansson}
\label{lisbeth_gunvor_maria_johansson}
Lisbeth föds 7 december 1956 på Lenhovda BB. Det finns inget BB i Åseda där familjen bor så Viva får åka taxi 3 mil mitt i vintern. Ensam, för på den här tiden är inte pappor välkomna att vara med under förlossningen.  Allt går bra och efter en vecka får mor och dotter åka hem till pappa Carl-Eric och storebröderna Bengt och Lars. Familjen bor på Södra Esplanaden i en tvårumslägenhet på tredje våningen. 
Jag minns inte den första semesterresan men det var en tältsemester som gick till Falkenberg när jag var 1½ år. Hur vi kom dit vet jag inte för detta var långt innan vi hade bil. Viva har berättat att det regnade och blåste mest hela tiden och att de hade för lite av allting med sig den gången.

Året därpå åkte vi till Kapelludden på Öland istället och där trivdes alla så dit åkte vi varje sommar tills jag var tonåring. Det krävdes noga planering inför resan, först packades allt som behövdes och skickades i förväg till stationen i Borgholm. Vi åkte med tåg till Kalmar, färja över sundet och till sist buss till Borgholm. Sedan fick vi bära all packning till campingplatsen, vilken som tur var låg nära stationen. Det måste ha varit tungt att få allt på plats för tält och all packning låg i en stor trälåda som min morbror Georg hade snickrat ihop. Dagarna fylldes med bad och lek. På kvällarna gick vi in till Borgholm, tittade på alla båtar i hamnen och åt mjukglass.

En vecka varje sommar tillbringade vi hos min morfar och morfar i Ädelfors. Mina tidigaste minnen är att jag satt i morfars knä vid skrivbordet och löste barnkorsordet, vi lekte och räknade småpengar. Mormor var för det mesta i köket och när hon tyckte att morfar sa något som hon inte höll med om så sa hon alltid ”Kära, vad Anders pratar.” 
Mormor lyssnade mycket på radion, hon tyckte om sången ” I natt jag drömde ” och varje gång den spelades på radion så skrev hon ner textbitar tills hon slut hade hela texten.

Mormor handlade alltid i Anderssons lanthandel, där man handlade ”över disk” dvs. man fick säga vad man skulle ha och sedan vägde och mätte fru Andersson upp det. De varor som inte fanns, beställdes från Vetlanda, enbärsdricka i kuttingar kom från Virserum.

Morbror Georg satt med i barnavårdsnämnden och vid ett möte om sommarbarn så blev han tillfrågad om han kunde ta emot en flicka från Stockholm. De som skulle tagit hand om henne hade sagt återbud. Georg och Ingrid bestämde sig för att den 7-åriga flickan Carmen fick komma till dem. Varje sommar , tills hon blev 12 år, tillbringade hon i Ädelfors. De sista somrarna var hennes systrar Soraya och Tekla också där. Carmen och jag hade mycket roligt dessa somrar, vi byggde ett stort filttält ute i trädgården och ibland åkte vi och badade i Byestadssjön. Carmen kunde simma, men bara under vatten, det gick inte alls när hennes huvud var över vattenytan.