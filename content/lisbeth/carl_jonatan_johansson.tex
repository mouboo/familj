\chapter{Carl Jonatan Johansson}
\label{carl_jonatan_johansson}
Lisbeths farfar, Carl Jonathan Johansson, född 17 januari 1895 i Stocksberg, flyttar till Bäck, Korsberga. Familjen består av föräldrarna Johan Peter och Marta Regina Johansson och fem äldre syskon. Carl blir faderlös vid 7-års-åldern. Pengar var det ont om och att köpa nya kläder till Carl hände sällan, han fick ärva det mesta av sina syskon , t.o.m flickkängor med klack av sin storasyster Ingrid.När han är 12 år arbetar han vid den kvarn-och sågverksrörelse vilken tidigare har ägts av hans far. Två år senare kom han i skräddarlära i sin hemsocken. Enligt C-E lärde sig Carl att sy hos skräddare Ekholm som troligtvis var alkoholist, för när han inte kunde få tag på sprit så åt han skokräm, eftersom en av ingredienserna var just sprit. 1915 genomgick Carl svenska tillskärar-och yrkeskolan och får sedan en anställning i Axel Karlssons skrädderiaffär i Åseda. 1923 åker han till Stockholm till J-E Kollbergs tillskärarskola. Hans titel blir skräddarmästare.Samma år startar han sin egen skrädderirörelse i Åseda. 1942 köper han en fastighet på Kexholmen, Åseda där han också har en kappaffär. Carl är en av initiativtagarna till Åseda Hantverks-och industriförening,tillhör styrelsen. Han är också med i Kronobergs län och Östra Smålands skräddar mästareförening.Ett annat uppdrag är ledamot i kyrkofullmäktige. Det sista uppdraget blir ordförandeskapet i Åseda Pensionärsförening, han avgår där 1970.

Den blå urnan som stod på vitrinskåpet i stora rummet hos Carl-Eric och Viva kommer från Åseda. Den har en liten historia: När farfar Carl hade skrädderi-och kappaffär så kom det in en dam från Stockholm och beställde två kappor, som skulle skickas till henne. Kapporna syddes och levererades till damen. Nu visar det sig att hon inte kunde betala dem med pengar utan hon skickar den blå urnan som betalning, hon ägde en antikvitetsaffär. Vi har i alla år trott att den skulle vara värdefull, men nu har vi värderat den och den är bara värd ca 400 kr. Det var ingen bra affär för farfar.

Carl hade 8 syskon, Emma Kristina Seraphia, f. 1876, Johan August f. 1878, Karl Stefanus f. 1881,  Ester Maria ”Maja” f. 1883, som faktiskt är något av en ”kändis”. Hon arbetade som lärarinna och en av hennes elever var författaren Vilhelm Moberg. Det var utav henne som han fick sänkt sedebetyg vilket han senare i livet skrev om i en bok. Men det var också hon som lärde honom att upptäcka böckernas värld. Knut Gunnar f. 1889, Ingrid Elisabet Kristina f. 1891, hon gifter sig med Emil Elvin som äger en pälsvaruaffär i Vetlanda. De säljer pälsar, hattar och mössor. Efter sin mans död 1927 så förestår hon affären själv.