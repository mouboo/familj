\chapter{Johanna Christina Lång}
\label{johanna_christina_lång}
Född på ett soldattorp tillhörigt gården Snuggarp (nuv.~Skönberga) i Skirö socken 1/7 1843. Gift med Johan Peter Cederblad från Sandbro, Repperda.
I äktenskapet föddes 11 barn. Känd för att vara något av ``klok gumma'', där både skrock och egenhändigt tillverkad örtmedicin ingick.

Hilma Cederblad, gift med sonen Anders, har berättat att här hon beklagade sig för att den några månader gamle sonen inte ville sova på nätterna utan skrek och var besvärlig, så sa hon att det kan du väl få något för. Så kom hon med en liten påse med okänt innehåll, som skulle hängas i ett band om halsen på pojken. Det skulle absolut hjälpa. När Hilma tvekade att använda ``medicinen'' så sa hon ``Är du så dum så du inte tror på det, så må du ha att [besväret].''

Blev mycket omtalad i hela bygden då hon lyckades bota en av stationsinspektor Axel Kjellins döttrar som drabbats av ``skerva'', engelska sjukan, med sin dekokt. De hade tidigare sökt läkarhjälp utan resultat.