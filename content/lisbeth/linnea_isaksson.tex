\chapter{Linnea Isaksson}
\label{linnea_isaksson}
Carls blivande fru, Linnèa är född 1894 i Gubbagården,i byn Galtabäck i Nottebäck socken. Föräldrarna heter Isak Pettersson och Stina Maria Johansson, de har sex barn.Gubbagården är Stina Marias föräldrahem som hon har fått överta. Isak är tyvärr inte så mån om gården och dess skötsel , till slut måste familjen gå ifrån den. Namnet Gubbagården kommer från talesättet att det var den gården som gubbarna samlades i. Carl-Erics minne av sin morfar är att han mest låg och sov på soffan. Hela familjen flyttar sedermera till Stenhuset i Klavreström. Linnéa kommer som ung flicka till Åseda 1914.Hon har fått anställning som kammarjungfru hos kyrkoherde Bergdahl. En dotter,Gullan Stéen, berättar följande historia: Linnéa är i parken och står vid dansbanan, lite längre bort står tre flickor och pratar. Hon hör att de pratar om henne, en av flickorna, Astrid Spjut,säger: Det är prästpigan. Linnéa tycker att det låter nedlåtande,men hon får en liten hämnd. En tid senare är hon i hattaffären, hon ser en hatt som hon genast tycker om men får veta av expediten att Astrid Spjut har tänkt köpa den. Hatten kostar 6 kronor (mycket pengar på den tiden) men Linnéa tar sig råd med tanken att ”Astrid Spjut ska inte få den”.
Linnéas föräldrar hade en tam berguv i en bur.

Linnéa och hennes syster Elna fick varsin pigtittare av sina föräldrar. Vid bodelningen efter Carl och Linnéa tog min faster Ann-Margret hand om den även att Carl-Eric hade blivit lovad av sin mor  att få den. Han bråkade inte med sin syster men var mycket ledsen att det blev som det blev.
Efter Elnas död hölls en auktion med bara hennes närmaste släktingar. När Elnas pigtittare kom upp försökte Ann-Margrets son Lennart ropa in den och än en gång var Carl-Eric tyst. Men då tyckte hans syster att det fick vara nog så hon bjöd emot och lyckades få det högsta budet, 500 kr. Efter auktionen gav hon pigtittaren till Carl-Eric och han betalade senare Gullan de 500 kronorna.
Pigtittaren är nu i min ägo. 
De två ekstolarna från Elnas hem har jag också.

Linnéa hade 7 syskon varav 2 dog i unga år. Äldste brodern Johan arbetade som vaktmästare vid Klavreströms Bruk. Han spelade flöjt i en orkester, han var intelligent och duktig i matematik. Han hjälpte Carl med hans bokföring. Storasyster Elna gifte sig med Sigurd Pettersson och bodde i Växjö. De hann inte vara gifta så många år. Utåt sätt var Elnas man den gladaste av alla, men en dag inträffade något som visade att allt inte stod rätt till. Paret skulle ta en promenad när Sigurd plötsligt säger att han har glömt något hemma och går tillbaka till hemmet. Elna väntar och väntar men han kommer inte. Hon går hem igen och när hon kommer innanför dörren så får hon se att han har hängt sig i trappan. Eftersom de hade levt på hans inkomst från sitt arbete som konduktör så blev livet hårt för Elna i dubbel bemärkelse. Hennes man är borta och hon har inga pengar till sitt uppehälle.
Men hon finner på råd. Hon tar emot inackorderingar, stryker tvätt åt andra och undervisar som skolhushållärare i sitt hem. Hon gifter aldrig om sig.
Nästa bror, Gustaf är slaktare. Lillebror Albert är predikant i Kristinehamn, Arboga och Malmö, han har t.o.m ett eget radioprogram där han håller sina predikningar. Han är gift med Ebba och har två barn, Gertrud och Sven. Det finns en historia om Sven och Carl-Eric, pojkarna hade lagt ut kräftburar och när det var dags att vittja dem, så hade de glömt att ta med något att förvara kräftorna i. De fick göra det bästa av situationen, så efterhand som de plockar upp kräftorna ur burarna så la de dem i sina kepsar. Det måste ha känts väldigt märkligt att ha levande kräftor krälande i håret  under kepsarna. Lillasyster Ottilia, ( som enligt Carl-Eric var en mycket högdragen person) gifter sig med en äldre man och bosätter sig i Värmland. Innan hon gifte sig arbetade hon på Stockholm slott.

Carl och Linnéa träffas troligen i prästgården. Linnéa arbetar ju där och Carl hyr ett rum hos kyrkoherden. Tycke uppstår och de gifter sig 1917. Deras första hem ligger på Karlavägen, vid bokhandeln bakom missionshuset. De får fyra barn, Maj-Britt, Ann-Margret, Carl-Eric och Gullan.
Senare köper de hus i Kexholm, där Carls skrädderi och kappaffär finns på nedervåningen.

Ann-Margrets son, Lennart, har berättat några minnen om Linnéa för mig. Hennes far var väldigt glad i sprit och gick gärna i god för hans vänner när de behövde pengar. Till slut gick det så illa att famljen blev tvungna att gå ifrån gården och flytta till Klavreström. Han var nog inte den bästa maken och pappa. För Linnéas mamma var det värst, gården var ju hennes barndomshem.
Med detta som bakgrund så fanns det ingen sprit i Carl och Linnéas hem, med ett undantag. I en kakelugn i ett sovrum som inte användes hade Linnéa en flaska konjak och när hon kände att en förkylning på gång eller liknande så hämtades flaskan och hon hällde upp konjak i en sked och tog den som medicin. Det var mycket viktigt att inte använda ett glas, för då var det inte medicin utan för nöjes skull. Det var Lennarts pappa Folke som fick hämta ut konjaken, varken Carl eller Linnéa tyckte att de kunde göra det själva.
Lennart har ett litet bord med en utdragslåda i sin ägo, det var Linnéas pappas matbord och i lådan hade han sina bestick. Vid detta bord åt han sina måltider ensam, resten av famljen fick stå och äta.