\chapter{Johanna Lovisa Johansdotter}
\label{johanna_lovisa_johansdotter}
Född 13/9 1845, i Godanstorp, Alseda socken.
Gifte sig 1/12 1871 med smeden Johan August Cecar, boende under Trollhälla Alseda socken.

Hemmafru. När Johan August gick bort övertogs smedjan av sonen Hjalmar Fredrik. När han gifte sig med Gulli Maria Teresia Franzén 1917 byggde han om och till stuga i Bärbäcksbron. Johanna fick dock bo kvar där några år. Men lönsamheten i smedjan blev allt sämre, och han måste söka arbete på annat håll. Stugan i Bärbäcksbron såldes och Johanna fick flytta till dottern Hilma Cederblad i Ädelfors. Efter en tid där flyttade hon till dottern Emma Karlssom i Slättåkra, där hon bodde några år innan hon fick sluta sina dagar 1930.