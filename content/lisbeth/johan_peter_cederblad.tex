\chapter{Johan Peter Cederblad}
\label{johan_peter_cederblad}
Föddes på husartorpet Sandbro, tillhörigt Repperda Sunegård, i Alseda socken, 19/1 1838.
Efter att ha läst för prästen började han som lärling hos en skräddare i Skirö. Någon skräddare blev han dock aldrig, men sydde till sina söner senare i livet. 

Han flyttade till Kibbe, troligen i samband med giftemålet med Johanna Christina Lång från Skirö. Där blev han bruksarbetare vid Kibbe Nickelverk, som drevs av Lessebro Bruk, med Bergsrådet Johan Lorentz Aschan som ägare.

Två eller tre somrar reste han till Stockholm och arbetade som byggnadsarbetare.
Det var kanske bättre betalt än arbetet vid bruket. Han fick då gå, eller kunde kanske få åka med någon forbonde som körde till Oskarshamn för att hämta varor. Därifrån åkte han med båt till Stockholm. Någon järnväg fanns ju inte på nära håll. Han hade också varit rallare vid Södra Stambanan.

Han fick 1896 Patriotiska Sällskapets medalj för långvarig trogen tjänst, i samband med att Lessebo Bruk AB lade ner sin verksamhet i Kibbe, och sålde sina gruvor och egendomar i Östra Härad, till en tysk konsul Bieber, som tidigare under tre år arrenderat och drivit Guldgruvan. Han bildade nu Ädelfors Guldverksaktiebolag för fortsatt drift av Ädelfors Guldgruva. 

Johan Peter fick ett årligt bidrag från ``Stenkvarnefonden'', en fond för gamla bruksarbetare, förmodligen startad av Lessebo Bruk AB, men förvaltades av Bergmästareämbetet, dit ansökan fick insändas.

Avled 1925.