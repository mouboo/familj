\chapter{Viva Gunhild Maria Cederblad}
\label{viva_gunhild_maria_cederblad}
Född 9 november 1920 i huset vid ”åbron” i Ädelfors. Föräldrarna hette Anders och Hilma Cederblad. Viva är efterlängtad och alla är glada att allt har gått bra. Två år tidigare föddes en dotter som fick namnet Berta, men hon blev bara fyra månader innan hon dog i influensa. Vivas bror,Georg, som då är 5 år minns Berta och sorgen som följde på hennes för tidiga död. Mor Hilma talade aldrig om Berta efter detta och därför gjorde ingen annan det heller.
I början på 1930-talet köper Anders och Hilma ett hus som får namnet Cederslund.  Anders renoverar det men det förblir med dagens mått mätt rätt så omodernt. Enbart kallvatten indraget, kakelugnar och vedspis som värmekälla och till matlagning. Utedass och och inget badrum. Kylskåp och frys fanns inte heller utan alla kylvaror förvarades in en matkällare. Så här lever familjen ända in på 1970-talet. Efterhand har ju barnen flyttat och när Anders dör 1970, så bor Hilma bara kvar ett par år i huset.
Skolhuset ligger bredvid Cederslund så Viva och hennes bröder behöver inte gå långt för att komma dit. Hon går i skolan i 6 år och efter det en skolkökskurs.

På 1930-talet, om man ville åka på semester och det inte fanns så mycket pengar, så åkte man på cykelsemester. Så gjorde även Viva och hennes vänner. Bl.a cyklade de till Tranås en sommar. Hennes bröder Georg och Henry, samt en kamrat, cyklade till Gripsholms slott och fortsatte till Trollhättan.Väl där skickade de ett vykort hem och skrev att de skulle ta en ”avstickare” till Göteborg innan de kom hem igen. De cyklade också till Kalmar, vidare till Karlskrona för att slutligen hamna i Ystad som var målet med resan. De skulle titta på en utställning där.
På sina cyklar hade de all utrustning som de behövde såsom tält,mat, fotogenkök, kläder m.m

1936 hände något som alla i Ädelfors talade om länge. Viva hade lämnat in en tipskupong och vann högsta vinsten. 15 040 kr och 67 öre. Det var en hel förmögenhet på den tiden. Hon hjälpte sina föräldrar att betala av lån på huset och bekostade sin egen utbildning till damfrisörska. Denna utbildning fanns i Vetlanda och troligtvis bodde hon där i veckorna. Hon kallades därefter för Lyckaflickan.

I slutet på 1930-talet kommer Viva Cederblad till Åseda. Hon har nu fått sitt första arbete på en salong i Åseda, vilken hon senare blir ägare till. Hennes bostad är en lägenhet ovanför salongen. Ibland var det nog inte så roligt att ta hand om kunderna. Det var sällsynt med damer som hade tvättat håret innan de kom för att bli fina. Det var inte ovanligt att Viva fick börja med att avlusa dem innan hon kunde börja klippa eller permanenta.

C-E och Viva träffas så småningom och blir ett par. Detta är under pågående 2:a världskriget så C-E blir inkallad som beredskapsman och blir stationerad i Skåne. Paret förlovar sig och innan bröllopet blir det björkdrag i Ädelfors. Detta är en väldigt gammal tradition som går ut på att brudparets vänner på lysningsdagen kommer dragandes på en stor björk och överlämnar den. Av björken ska sedan brudgummen tillverka någon möbel till det framtida egna hemmet. Det blev Vivas bror Georg som svarvade ljusstakar och nötknäckare av björken.
23 oktober 1943 vigdes C-E och Viva i prästgården i Åseda av prosten d:r I. Krook. Brudpar och släkt åkte sedan till Ädelfors för bröllopsmiddag hos Vivas föräldrar Anders och Hilma Cederblad. Faster Gullan minns att Vivas bror Henry spelade dragspel. Vid hemkomsten till Vivas lägenhet så blir brudparet och den närmsta släkten bjudna på kaffe av Vivas granne, Gerda Olsson.

Efter vigseln måste Viva sälja sin salong och börja sitt nya liv som gift. Det var ovanligt att kvinnan fortsatte att arbeta när hon hade gift sig.

Vissa omständigheter gör att deras första tid som gifta blir omvälvande för båda två. Viva är gravid i femte månaden och måste nu sälja sin damfrisering och flytta från sin lägenhet. Carl och Linnéa ger dem husrum i sitt hus, ett rum och kök i en tillbyggnad mot trädgården. C-E får inte tillräckligt med lön av sin far för att kunna försörja sin familj utan måste låna av Vivas sparade pengar. De lägger upp en avbetalningsplan som C-E lyckas hålla. I mars 1944 föds Bengt och det dröjer ända till maj 1952 innan nästa barn, Lars, utökar familjen. Någon gång mellan 1952 och 1956 flyttar de till en alldeles ny hyreslägenhet på Södra Esplanaden i Åseda. I december 1956 föds en dotter Lisbeth.
Slutet av 1964 har C-E fått ett arbete som förman på Bestons konfektionsfabrik i Berga och flytten går denna gång till Högsby. 

Arvegods från Carl-Eric och Viva: guldringen med en liten vit sten fick mamma av pappa en gång för länge sedan. Hon hade den alltid mellan sin förlovnings- och vigselring. Glasskålen på fot fick Anders och Hilma i lysningspresent 1913. Silverbesticken köpte hon i en affär i Åseda som hette Dacke. Finservisen som min bror Lars har nu är också inköpt där. Taklampan som jag har är köpt i Åseda. 
Stekgrytan med mammas namn och ett årtal på locket är en lysningspresent från min farmors syskon. Den är tillverkad i Klavreströms Bruk.
Halsbandssmycket i form av en gulddroppe som Jenny har är ursprungligen min farmors ringar.


Vivas syskon:

Georg Cederblad, född 24/9 1913 i Smedstugan, Ädelfors. Gift med Ingrid. Två barn, Åke och Kristina.
Georg första arbeten får han i skogen och på ett sågverk i Ädelfors. 1935-1948 arbetar han i Källströms möbelfabrik och sedan hos Bröderna Granlund. 1952 till sin pensionering i Hermanssons möbelfabrik i Fluguby.
Hans fru, Ingrid, arbetar som sömmerska i en syateljé som ägs av Sjöberg. 
Georg byggde hus mittemot Cederslund och kallade det för Lillevång, (nästan alla hus i Ädelfors har ett namn).

Henry Cederblad, född 29/6 1916 i Smedstugan, Ädelfors. Gift med Inga.
Han arbetar liksom som sin bror i Hermanssons möbelfabrik i Fluguby.
Hans fru, Inga, övertog tjänsten som stationsföreståndare 1934 på Tällängs station efter sin far, Oskar Tapper. Han hade dock ansvaret för stationen tills Inga blev myndig. 1937 får Oskar nytt arbete på Ädelfors station där han arbetar till 1943. Inga övertar hans arbete och blir kvar till 1961. Poststationen dras in då persontrafiken läggs ner på sträckan Vetlanda-Gårdveda. Inga förflyttas till Alseda station och arbetar där som platsvakt fram till 1966 då också den poststationen dras in. Hon får anställning på posten i Vetlanda där hon är kvar till sin pension.